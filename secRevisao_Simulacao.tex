%TCIDATA{LaTeXparent=0,0,cap_RevisaoBibliografica.tex}

\section{Simula\c{c}\~{a}o Num\'{e}rica de Reservat\'{o}rios}

A simula\c{c}\~{a}o num\'{e}rica de um reservat\'{o}rio \'{e}, segundo Peaceman, \'{e} o processo de infer\^{e}ncia do comportamento do reservat\'{o}rio real dada a performance obtida de um modelo do mesmo, matem\'{a}tico ou f\'{i}sico (em escala laboratorial). Um modelo matem\'{a}tico de reservat\'{o}rio pode ser enxergado como um conjunto de equa\c{c}\~{o}es diferenciais parciais, juntamente com as condi\c{c}\~{o}es de contorno adequadas, que podem ser utilizadas para descrever satisfatoriamente os processos f\'{i}sicos importantes que ocorrem no sistema real.

Os processos que ocorrem em um reservat\'{o}rio s\~{a}o basicamente transporte de fluidos e transfer\^{e}ncia de massa; at\'{e} tr\^{e}s fases imisc\'{i}veis (\'{o}leo, g\'{a}s e \'{a}gua) fluem simultaneamente, enquanto que o transporte de massa se d\'{a} entre as fases (notadamente entre o \'{o}leo e o g\'{a}s). A gravidade, a capilaridade e as for\c{c}as viscosas s\~{a}o tamb\'{e}m importantes no processo de vaz\~{a}o dos fluidos \cite{simres}.

Segundo Rosa, a primeira etapa de uma simula\c{c}\~{a}o num\'{e}rica \'{e} formular o problema f\'{i}sico a ser representado matematicamente; em seguida s\~{a}o feitas suposi\c{c}\~{o}es e simplifica\c{c}\~{o}es compat\'{i}veis com o grau de sofistica\c{c}\~{a}o esperado do modelo, levando-se \`{a} formula\c{c}\~{a}o das equa\c{c}\~{o}es matem\'{a}ticas que descrevem o problema desejado, considerando-se as hip\'{o}teses adotadas. O passo seguinte \'{e} a resolu\c{c}\~{a}o das equa\c{c}\~{o}es e an\'{a}lise da solu\c{c}\~{a}o obtida; posteriormente, a validade do simulador \'{e} verificada atrav\'{e}s da calibra\c{c}\~{a}o com uma solu\c{c}\~{a}o existente --- por exemplo, comparam-se os resultados obtidos do simulador num\'{e}rico com solu\c{c}\~{o}es anal\'{i}ticas, resultados reais ou com resultados obtidos de modelos f\'{i}sicos de laborat\'{o}rio (dados experimentais). Caso o simulador seja considerado v\'{a}lido, o mesmo estar\'{a} pronto para ser utilizado na simula\c{c}\~{a}o do fen\^{o}meno desejado; caso contr\'{a}rio, volta-se para um novo ciclo em que s\~{a}o revistas as hip\'{o}teses adotadas ou at\'{e} a conceitua\c{c}\~{a}o do modelo f\'{i}sico \cite[p. 520]{engres}. A Figura \ref{fig:rev_simuesq} esquematiza um desenvolvimento b\'{a}sico de um simulador num\'{e}rico qualquer, enquanto que a Figura \ref{fig:rev_simuex} mostra uma compara\c{c}\~{a}o de resultados entre diferentes simuladores existentes, exemplificando o uso da calibra\c{c}\~{a}o com solu\c{c}\~{o}es j\'{a} obtidas para se validar um simulador de reservat\'{o}rio. 

\begin{figure}[H]
\centering
\includegraphics[width=.75\textwidth]{figs/revisao/revisao_simuesq.png}
\caption{Esquema b\'{a}sico de desenvolvimento de um simulador num\'{e}rico de reservat\'{o}rio \cite[p. 519]{engres}.}\label{fig:rev_simuesq}
\end{figure}

\begin{figure}[H]
\centering
\includegraphics[width=.75\textwidth]{figs/revisao/revisao_simuex.png}
\caption{Exemplo de compara\c{c}\~{a}o de dados entre simuladores de vaz\~{a}o de \'{a}gua e de \'{o}leo de um modelo \cite{eggM}.}\label{fig:rev_simuex}
\end{figure}

\subsection{Leis F\'{i}sicas Consideradas}

No caso de um simulador de reservat\'{o}rios, as seguintes leis f\'{i}sicas b\'{a}sicas normalmente s\~{a}o consideradas, dependendo do tipo de simulador\footnote{Ver \cite[p. 520]{engres}}:

\begin{itemize}
\item Lei da conserva\c{c}\~{a}o de massa;
\item Lei da conserva\c{c}\~{a}o de energia;
\item Lei da conserva\c{c}\~{a}o de \textit{``momentum''} (Segunda Lei de Newton):
\begin{equation}
\sum F = \frac{\partial M}{\partial t},
\end{equation}
onde $F$ representa uma for\c{c}a e $M = mv$ o \textit{``momentum''}, com $m$ sendo a massa e $v$ a velocidade.
\end{itemize}

Al\'{e}m das leis b\'{a}sicas da f\'{i}sica, faz-se necess\'{a}rio o uso de v\'{a}rias leis, dependendo do simulador, que governam o comportamento dos fluidos envolvidos e a propriedade do reservat\'{o}rio estudado, apresentadas nas subse\c{c}\~{o}es a seguir\footnote{Os teoremas apresentados se encontram em \cite[pp. 520-522]{engres}}. Combinado-se as equa\c{c}\~{o}es correspondentes \`{a}s leis b\'{a}sicas, obt\'{e}m-se uma equa\c{c}\~{a}o diferencial parcial que rege o comportamento das vari\'{a}veis dependentes em fun\c{c}\~{a}o das vari\'{a}veis independentes e dos par\^{a}metros do sistema. Como normalmente a equa\c{c}\~{a}o obtida \'{e} n\~{a}o-linear, ela \'{e}, consequentemente, \'{e} resolvida por m\'{e}todos n\'{u}mericos; da\'{i} a nomenclatura \textit{simula\c{c}\~{a}o num\'{e}rica de reservat\'{o}rios}.

\subsubsection{Fen\^{o}menos de Transporte}

\begin{theorem}[Lei de Darcy]
Na Lei de Darcy, ou lei do fluxo ``laminar'' ou Darcyano, a velocidade do fluxo viscoso de um fluido em meio poroso \'{e} dada por
\begin{equation}
	v_s = -\frac{k_s}{\mu} \frac{\partial\Phi}{\partial s},
\end{equation}
onde $k$ \'{e} a permeabilidade efetiva do meio ao fluido considerado, $\mu$ \'{e} a viscosidade do fluido, $\Phi$ \'{e} o potencial de fluxo e $s$ \'{e} a trajet\'{o}ria de fluxo.
\end{theorem}

\begin{theorem}[Lei de Forchheimer]
Tamb\'{e}m conhecida como lei do fluxo ``turbulento'' ou n\~{a}o-Darcyano, \'{e} utilizada para fluxos turbulentos, notadamente de g\'{a}s; o gradiente de press\~{a}o \'{e} dado por
\begin{equation}
	-\frac{dp}{ds} = \frac{\mu}{k_s}v_s - \beta\rho v_s^2,
\end{equation}
onde $\rho$ \'{e} a massa espec\'{i}fica do fluido e $\beta$ \'{e} o coeficiente de resist\^{e}ncia inercial ou de fluxo n\~{a}o-Darcyano. 
\end{theorem}

\begin{theorem}[Lei de Fourier]
Durante um fen\^{o}meno de transporte de calor por condu\c{c}\~{a}o, o fluxo de calor \'{e} dado por
\begin{equation}
	q_s = -k'\frac{\partial T}{\partial s},
\end{equation}
em que $k'$ \'{e} a condutividade t\'{e}rmica do meio e $T$ \'{e} a temperatura.
\end{theorem}

\begin{theorem}[Convec\c{c}\~{a}o]
O fluxo de calor no caso de tranporte por convec\c{c}\~{a}o \'{e} dado por
\begin{equation}
	q_s = c_p v_s (T - T_0),
\end{equation}
onde $c_p$ \'{e} a capacidade calor\'{i}fica do fluido \`{a} press\~{a}o constante, $v$ a velocidade do fluido e $T_0$ uma temperatura de refer\^{e}ncia.
\end{theorem}

\subsubsection{Equa\c{c}\~{o}es de Estado}
As principais equa\c{c}\~{o}es de estado envolvidas na simula\c{c}\~{a}o do comportamento de um reservat\'{o}rio de petr\'{o}leo s\~{a}o as que lidam com fluidos (l\'{i}quidos ou gasosos) e rochas porosas. No caso de fluidos l\'{i}quidos, tem-se a seguinte defini\c{c}\~{a}o:

\begin{definition}
A compressibilidade isot\'{e}rmica de um fluido \'{e} dada por
\begin{equation}
	c = -\frac{1}{V}\frac{\partial V}{\partial p} = \frac{1}{\rho}\frac{\partial \rho}{\partial p},
\end{equation}
em que $V$ \'{e} o volume, $p$ \'{e} a press\~{a}o e $\rho$ \'{e} a massa espec\'{i}fica do fluido. H\'{a} algumas rela\c{c}\~{o}es especiais para situa\c{c}\~{o}es particulares:
\begin{itemize}
\item L\'{i}quidos de compressibilidade constante: $\rho = \rho_0 e^{c(p-p_0)}$.
\item L\'{i}quidos de compressibilidade constante e pequena: $\rho = \rho_0 \left[1+c\left(p-p_0\right)\right]$.
\end{itemize}
\end{definition}

Quando se trata de estudar o estado de um g\'{a}s, se aplica a lei dos gases:
\begin{equation}\label{eq:gaslaw}
	\rho = \frac{pM}{ZRT}.
\end{equation}

A Equa\c{c}\~{a}o \eqref{eq:gaslaw} pode ser aplicada tanto no caso de um g\'{a}s real quanto de um gaś ideal; nela, $\rho$ \'{e} a massa espec\'{i}fica do g\'{a}s, $p$ \'{e} a press\~{a}o, $M$ \'{e} a massa molecular, $R$ \'{e} a constante universal dos gases, $T$ \'{e} a temperatura e $Z$ \'{e} o fator de compressibilidade do g\'{a}s; no caso de um g\'{a}s ideal, tem-se $Z = 1$.

Por fim, para se representar o comportamento da rocha, utiliza-se a equa\c{c}\~{a}o da chamada compressibilidade efetiva:
\begin{equation}
	c_f = \frac{1}{\phi} \frac{\partial\phi}{\partial p},
\end{equation}
onde $c_f$ \'{e} a compressibilidade efetiva efetiva da forma\c{c}\~{a}o e $\phi$, sua porosidade.

Al\'{e}m das leis at\'{e} aqui citadas, cabe ressaltar que outras podem ser utilizadas em caso de simula\c{c}\~{o}es de fen\^{o}menos espec\'{i}ficos, como inje\c{c}\~{a}o de vapor, inje\c{c}\~{a}o de pol\'{i}meros, al\'{e}m de outros m\'{e}todos empregados na produ\c{c}\~{a}o de petr\'{o}leo.

\subsection{Tipos de Simuladores}
Segundo Rosa, os simuladores de reservat\'{o}rios podem ser classificados em fun\c{c}\~{a}o de tr\^{e}s crit\'{e}rios b\'{a}sicos: o tratamento matem\'{a}tico utilizado, o n\'{u}mero de dimens\~{o}es consideradas e o n\'{u}mero de fases admitidas. Em rela\c{c}\~{a}o \`{a} matem\'{a}tica do simulador, os simuladores podem ser classificados em: 

\begin{itemize}
	\item \textbf{Modelo Beta ou volum\'{e}trico:} \'{e} tamb\'{e}m conhecido como \textit{black oil}; nesse modelo, s\~{a}o consideradas as fun\c{c}\~{o}es de press\~{a}o e da temperatura do reservat\'{o}rio. Al\'{e}m disso, cada fase presente no reservat\'{o}rio (\'{a}gua, \'{o}leo e/ou g\'{a}s) \'{e} admitida como constitu\'{i}da por apenas um componente, mesmo que, na pr\'{a}tica, o \'{o}leo seja composto por v\'{a}rios hidrocarbonetos, al\'{e}m de impurezas.
	\item \textbf{Modelo composicional:} Al\'{e}m de considerar a press\~{a}o e a temperatura do reservat\'{o}rio, tamb\'{e}m se admite as composi\c{c}\~{o}es das diversas fases que estejam presentes no meio poroso. Ao contr\'{a}rio do \textit{black oil}, por exemplo, o \'{o}leo passa a ser tratado pelos seus v\'{a}rios hidrocarbonetos de que \'{e} composto, tais como $C_1$, $C_2$, $C_3$, etc. Por\'{e}m, como o n\'{u}mero de componentes no \'{o}leo \'{e} grande, alguns hidrocarbonetos s\~{a}o agrupados nos chamados \textit{pseudocomponentes}; a utiliza\c{c}\~{a}o dessa abordagem reduz o tempo computacional necess\'{a}rio ao modelo, uma vez que um tratamento mais rigoroso poderia tornar impratic\'{a}vel a simula\c{c}\~{a}o composicional.
	\item \textbf{Modelo t\'{e}rmico:} \'{e} utilizado quando \'{e} necess\'{a}rio considerar os efeitos de varia\c{c}\~{o}es t\'{e}rmica no interior do reservat\'{o}rio --- por exemplo, quando se estuda a aplica\c{c}\~{a}o de m\'{e}todos t\'{e}rmicos de recupera\c{c}\~{a}o secund\'{a}ria, como inje\c{c}\~{a}o de vapor, inje\c{c}\~{a}o de \'{a}gua quente ou combust\~{a}o \textit{in situ}. Como os modelos t\'{e}rmicos tratam situa\c{c}\~{o}es complexas, eles s\~{a}o necessariamente composicionais.
\end{itemize}

Quanto ao n\'{u}mero de dimens\~{o}es, os simuladores s\~{a}o classificados de acordo com o n\'{u}mero de dimens\~{o}es nas quais se admite fluxo. Neste sentido, eles podem ser classificados em \textit{unidimensionais} (Figura ...), \textit{bidimensionais} (Figura ...) e \textit{tridimensionais} (Figura ...). Por fim, os simuladores num\'{e}ricos podem ser classificados de acordo com o n\'{u}mero de fases: \textit{monof\'{a}sicos}, caso haja apenas uma fase (no caso de \'{a}gua, se trata de um aqu\'{i}fero); \textit{bif\'{a}sicos}, quando h\'{a} duas fases presentes (\'{a}gua e \'{o}leo, no caso de reservat\'{o}rios de \'{o}leo, ou \'{a}gua e g\'{a}s, nos reservat\'{o}rios de g\'{a}s); e \textit{trif\'{a}sicos}, no caso da exist\^{e}ncia de tr\^{e}s fases (\'{a}gua, \'{o}leo e gas)\footnote{A classifica\c{c}\~{a}o dos modelos de simula\c{c}\~{a}o podem ser encontradas em \cite[pp. 517--519]{engres}}.

%TODO : Insira as figuras AQUI!

\'{e} importante que, ao se escolher um simulador num\'{e}rico para se resolver problemas de engenharia de reservat\'{o}rio, se considere v\'{a}rios fatores, a saber: o tipo de estudo a ser feito, tipo e caracter\'{i}sticas do reservat\'{o}rio e dos fluidos presentes, quantidade e qualidade dos dados, o detalhamento necess\'{a}rio do estudo e os recursos computacionais dispon\'{i}veis \cite[p. 519]{engres}. Por exemplo, \'{e} impratic\'{a}vel o uso de modelos composicionais em computadores cuja capacidade seja compar\'{a}vel a um computador pessoal de alto desempenho, devido \`{a} complexidade dos c\'{a}lculos envolvidos. Por outro lado, por sua simplicidade, um modelo \textit{black oil} poderia ser considerado, respeitando-se ao m\'{a}ximo as caracter\'{i}sticas do reservat\'{o}rio estudado.

\subsection{Uso de Simuladores Num\'{e}ricos para Estudos de Reservat\'{o}rios}

O uso de simuladores num\'{e}ricos torna poss\'{i}vel analisar o comportamento de um reservat\'{o}rio ao longo do tempo, dado um esquema de produ\c{c}\~{a}o. Dessa forma, pode-se obter, por exemplo, as condi\c{c}\~{o}es \'{o}timas de produ\c{c}\~{a}o, al\'{e}m de se determinar como a inje\c{c}\~{a}o de diferentes tipos de fluidos ou outros m\'{e}todos de EOR afetam o sistema simulado, determinar o efeito da localiza\c{c}\~{a}o dos po\c{c}os na recupera\c{c}\~{a}o de \'{o}leo e/ou g\'{a}s e analisar a influ\^{e}ncia de diferentes vaz\~{o}es de produ\c{c}\~{a}o e/ou inje\c{c}\~{a}o. O simulador obt\'{e}m seus resultados de informa\c{c}\~{o}es de natureza geol\'{o}gica, propriedades da rocha e dos fluidos presentes no meio poroso, hist\'{o}ricos de produ\c{c}\~{a}o (vaz\~{o}es e/ou produ\c{c}\~{o}es acumuladas de \'{o}leo e \'{a}gua) e de press\~{a}o, e outras informa\c{c}\~{o}es a respeito dos po\c{c}os de petr\'{o}leo, assim como as caracter\'{i}sticas de completa\c{c}\~{a}o \cite[pp. 522--523]{engres}. A Figura \ref{fig:revisao_simsec1} ilustra a aplica\c{c}\~{a}o de simuladores num\'{e}ricos para engenharia de reservat\'{o}rios.

\begin{figure}[!ht]
	\centering
	\includegraphics[width=.75\textwidth]{figs/revisao/revisao_simsec1}
	\caption{Aplica\c{c}\~{a}o de simuladores num\'{e}ricos em reservat\'{o}rios \cite[p. 522]{engres}}
	\label{fig:revisao_simsec1}
\end{figure}  

As etapas normalmente seguidas durante a simula\c{c}\~{a}o num\'{e}rica de um reservat\'{o}rio s\~{a}o \footnote{Ver \cite[pp. 523--524]{engres}.}:

\begin{enumerate}
\item \textbf{Coleta e prepara\c{c}\~{a}o dos dados:} \'{e} a fase de armazenamento e interpreta\c{c}\~{a}o de todos os dados cab\'{i}veis ao problema, sejam eles geol\'{o}gicos, propriedades da rocha e dos fluidos, entre outros. Quanto maiores a quantidade e a qualidade desses dados, mais confi\'{a}vel ser\'{a} a simula\c{c}\~{a}o.
\item \textbf{Prepara\c{c}\~{a}o do modelo num\'{e}rico:} Ocorre logo ap\'{o}s a tomada dos dados. Inicialmente, \'{e} feito o \textit{lan\c{c}amento do} grid \textit{ou malha}, onde \'{e} constru\'{i}da uma malha para se transpor as informa\c{c}\~{o}es necess\'{a}rias para o modelo. Logo, \'{e} feita a divis\~{a}o do reservat\'{o}rio em v\'{a}rias c\'{e}lulas, cada uma funcionando como um reservat\'{o}rio menor, conforme mostra a Figura \ref{fig:revisao_simsec3}.
\begin{figure}[H]
	\centering
	\includegraphics[width=.5\textwidth]{figs/revisao/revisao_simsec3}
	\caption{Malha utilizada na simula\c{c}\~{a}o num\'{e}rica de um reservat\'{o}rio \cite[p. 524]{engres}}
	\label{fig:revisao_simsec3}
\end{figure}
\item \textbf{Ajuste de hist\'{o}rico:} O objetivo desta etapa \'{e} calibrar o modelo num\'{e}rico com o reservat\'{o}rio real a partir dos melhores dados dispon\'{i}veis referentes aos hist\'{o}ricos de produ\c{c}\~{a}o e de press\~{a}o. O ajuste de hist\'{o}rico \'{e} um c\'{a}lculo do comportamento passado do reservat\'{o}rio e a consequente compara\c{c}\~{a}o com o hist\'{o}rico do campo ou do mesmo reservat\'{o}rio. Se a concord\^{a}ncia n\~{a}o \'{e} satisfat\'{o}ria, s\~{a}o necess\'{a}rios ajustes nos dados at\'{e} se obter resultados adequados. De todo modo, a import\^{a}ncia de se obter um bom ajuste de hist\'{o}rico reside no fato de que o modelo poder\'{a} ser utilizado para se efetuar previs\~{o}es confi\'{a}veis em rela\c{c}\~{a}o ao seu comportamento futuro.
\item \textbf{Extrapola\c{c}\~{a}o:} Uma vez que o ajuste de hist\'{o}rico \'{e} realizado, procede-se \`{a} fase de extrapola\c{c}\~{a}o, isto \'{e}, a previs\~{a}o de comportamento futuro do modelo. Podem ser impostas vaz\~{o}es e press\~{o}es para todos os po\c{c}os, condi\c{c}\~{o}es dessas vaz\~{o}es, entre outros. Essa etapa permite avaliar v\'{a}rios esquemas de produ\c{c}\~{a}o, e seus resultados podem ser utilizados em avalia\c{c}\~{o}es econ\^{o}micas, tornando poss\'{i}vel decidir pelo esquema \'{o}timo de produ\c{c}\~{a}o.
\end{enumerate}

Todas as etapas de simula\c{c}\~{a}o num\'{e}rica de reservat\'{o}rios est\~{a}o resumidos na Figura \ref{fig:revisao_simsec2}.

\begin{figure}[!ht]
	\centering
	\includegraphics[width=.75\textwidth]{figs/revisao/revisao_simsec2}
	\caption{Etapas da simula\c{c}\~{a}o num\'{e}rica de um reservat\'{o}rio \cite[p. 523]{engres}}
	\label{fig:revisao_simsec2}
\end{figure}