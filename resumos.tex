%TCIDATA{LaTeXparent=0,0,relatorio.tex}

\chapter[RESUMO]{RESUMO}
{O presente trabalho busca apresentar e prover uma implementa\c{c}\~{a}o inicial de um algoritmo alternativo de controle de produ\c{c}\~{a}o em reservat\'{o}rios, al\'{e}m de analisar seus resultados iniciais. O algoritmo desenvolvido, nomeado \textit{Smart Reservoir}, foi concebido com a inten\c{c}\~{a}o de obter uma rentabilidade equivalente aos valores encontrados na literatura com custo computacional e tempo de produ\c{c}\~{a}o reduzidos, partindo da premissa de que o dinheiro perde valor ao longo do tempo. 

A solu\c{c}\~{a}o proposta para o problema de se antecipar a produ\c{c}\~{a}o de petr\'{o}leo foi testada em dois modelos de reservat\'{o}rio bastante conhecidos: o \textit{Egg Model} e o SAIGUP. A premissa do \textit{Smart Reservoir} \'{e}, dada uma fun\c{c}\~{a}o objetivo bastante utilizada para otimiza\c{c}\~{a}o de produ\c{c}\~{a}o --- NPV---, modific\'{a}-la de maneira a se considerar na an\'{a}lise par\^{a}metros de reservat\'{o}rio al\'{e}m dos dados de produ\c{c}\~{a}o; assim, torna se poss\'{i}vel comparar diferentes estrat\'{e}gias de simula\c{c}\~{a}o utilizando-se apenas simula\c{c}\~{o}es de reservat\'{o}rio de m\'{e}dio termo, promovendo uma redu\c{c}\~{a}o significativa de custos computacionais. Al\'{e}m dessa redu\c{c}\~{a}o, que chega a cerca de 90 \%,  \'{e} poss\'{i}vel obter uma redu\c{c}\~{a}o de tempo de produ\c{c}\~{a}o de pelo menos 50 \%, sem prejudicar os resultados econ\^{o}micos obtidos por meio de outros m\'{e}todos de otimiza\c{c}\~{a}o presentes na literatura.

\textbf{Palavras-chave:} Otimiza\c{c}\~{a}o, Reservat\'{o}rio, Simula\c{c}\~{a}o, Redu\c{c}\~{a}o de Custos, NPV
}

\vspace*{2cm}

\newpage

\chapter[ABSTRACT]{ABSTRACT}
{Abstract.}