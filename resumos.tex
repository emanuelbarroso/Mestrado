%TCIDATA{LaTeXparent=0,0,relatorio.tex}

\chapter[RESUMO]{RESUMO}
{O presente trabalho busca apresentar e prover uma implementa\c{c}\~{a}o inicial de um algoritmo alternativo de controle de produ\c{c}\~{a}o em reservat\'{o}rios de petr\'{o}leo, al\'{e}m de analisar seus resultados iniciais. O algoritmo desenvolvido, nomeado \textit{Smart Reservoir}, foi concebido com a inten\c{c}\~{a}o de obter uma rentabilidade equivalente aos valores encontrados na literatura com custo computacional e tempo de produ\c{c}\~{a}o reduzidos, partindo da premissa de que o dinheiro perde valor ao longo do tempo. 

A solu\c{c}\~{a}o proposta para o problema de se antecipar a produ\c{c}\~{a}o de petr\'{o}leo foi testada utilizando-se dois modelos de reservat\'{o}rio conhecidos na ind\'{u}stria petrol\'{i}fera: o \textit{Egg Model} e o SAIGUP. A ideia central do \textit{Smart Reservoir} \'{e}, dada uma fun\c{c}\~{a}o objetivo bastante utilizada para otimiza\c{c}\~{a}o de produ\c{c}\~{a}o --- NPV---, modific\'{a}-la de maneira a se considerar na an\'{a}lise par\^{a}metros de reservat\'{o}rio al\'{e}m dos dados de produ\c{c}\~{a}o; assim, torna se poss\'{i}vel comparar diferentes estrat\'{e}gias de simula\c{c}\~{a}o utilizando-se apenas simula\c{c}\~{o}es de reservat\'{o}rio de m\'{e}dio termo, promovendo uma redu\c{c}\~{a}o significativa de custos computacionais. Al\'{e}m dessa redu\c{c}\~{a}o, que pode passar de 95\%,  \'{e} poss\'{i}vel obter uma redu\c{c}\~{a}o de tempo de produ\c{c}\~{a}o de pelo menos 50\%, sem prejudicar os resultados econ\^{o}micos obtidos por meio de outros m\'{e}todos de otimiza\c{c}\~{a}o presentes na literatura.

\textbf{Palavras-chave:} Otimiza\c{c}\~{a}o, Reservat\'{o}rio, Simula\c{c}\~{a}o, Redu\c{c}\~{a}o de Custos, NPV
}

\vspace*{2cm}

\newpage

\chapter[ABSTRACT]{ABSTRACT}
{
The present work aims to present and provide an initial implementation of an alternative control algorithm of production in oil reservoirs, besides analyzing its initial results. The developed algorithm, named Smart Reservoir, was conceived with the intention of obtaining an equivalent profit to the values found in literature with reduced computational cost and production time, starting from the premise that money loses its value over time.

The proposed solution for the problem of anticipating oil production was tested using two reservoir models acknowledged in oil industry: the Egg Model and the SAIGUP. The central idea of the Smart Reservoir is, given an objective function fairly used in production optimization -- NPV --, to modify it in order to consider in the analysis reservoir parameters in addition to the production data. Therefore, it becomes possible to compare different simulation strategies using only medium term reservoir simulations, thus promoting a significant reduction of computational costs. Besides this reduction, which can go over 95\%, it is possible to obtain a production time reduction of at least 50\% without harming the economical results obtained by other optimization methods present in literature.
 
\textbf{Key words:} Optimization, Reservoir, Simulation, Cost Reduction, NPV
}