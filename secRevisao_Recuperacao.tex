%TCIDATA{LaTeXparent=0,0,cap_RevisaoBibliografica.tex}

\section{Métodos Convencionais de Recuperação Secundária}

\subsection{Conceito e Contextualização da Recuperação Secundária}

De acordo com Rosa \textit{et al.}, nas acumulações de petróleo há, na época de sua descoberta, uma dada quantidade de energia, chamada de \textit{energia primária}, cuja grandeza é determinada pelo volume e pela natureza dos fluidos existentes no meio, além dos níveis de pressão e temperatura do reservatório. Quando se dá o processo de produção, parte dessa energia é dissipada por causa da descompressão dos fluidos do reservatório e das resistências que os mesmos encontram ao fluir em direção aos poços produtores --- resistências associadas às forças viscosas e capilares presentes no meio poroso. A consequência dessa dissipação de energia primária resulta no decréscimo de pressão do reservatório em sua vida produtiva e, consequentemente, na redução da produtividade dos poços. A quantidade de óleo retirada utilizando-se unicamente a energia do reservatório é denominada \textit{recuperação primária}.

De forma a se minorar os efeitos danosos da dissipação da energia primária, existem duas linhas de ação a serem consideradas:

\begin{itemize}
\item Reduzir as resistências viscosas e/ou capilares por meio de métodos especiais, como por exemplo aquecendo a jazida;
\item Adicionar suplemento de energia secundária, artificialmente comunicada, através de injeção de fluidos em poços selecionados.
\end{itemize}

Quando se suplementa o reservatório com energia transferida artificialmente, ou se empregam meios de incrementar a eficiência da energia primária, a quantidade adicional de óleo produzida é chamada de \textit{recuperação secundária}. Por extensão, todas as operações que conduzem à obtenção desse adicional de óleo também são denominadas recuperação secundária. Essas operações, atualmente, são implantadas em sua grande maioria o tão cedo quanto possível na vida do reservatório.

É importante ressaltar que há uma diferença entre recuperação secundária e métodos de elevação artificial e de estimulação de poços; estes não afetam diretamente as energias expulsivas do reservatório, embora sua aplicação concorra para economizá-las. As técnicas de elevação artificial e de estimulação de poços estão mais ligadas ao comportamento dos poços produtores do que ao comportamento do reservatório como um todo. Contudo, a linha divisória entre tais métodos e os métodos de recuperação secundária não é muito nítida --- certos métodos de estimulação, como a injeção cíclica de vapor, são usualmente incluídos entre os métodos de recuperação secundária \cite{engres}.

Ainda segundo Rosa \textit{et al.}, há dois objetivos práticos básicos dos métodos de recuperação secundária:

\begin{itemize}
\item \textbf{Aumento da eficiência de recuperação} --- A eficiência de recuperação primária é normalmente baixa; em alguns casos, dependendo das características do reservatório e dos fluidos, ela pode ser até nula. Em alguns casos, a eficiência de recuperação secundária pode passar de 60\% em casos bem-sucedidos; contudo, o valor mais frequente dessa eficiência, nos métodos convencionais, se situa entre 30 e 50\%.
\item \textbf{Aceleração da produção} --- O emprego dos métodos de recuperação secundária busca acelerar a produção ou ao menos reduzir a taxa de seu declínio natural. A aceleração da produção resulta em antecipação do fluxo de caixa; portanto, há o aumento de seu valor presente e uma consequente melhoria da economicidade da exploração do campo ou reservatório.
\end{itemize}

Além dos objetivos básicos de emprego da recuperação secundária, Rosa \textit{et al.} citam vários incentivos ao uso desses métodos, tais como: preço do petróleo; custos de exploração, desenvolvimento e produção; e avanços tecnológicos na área. Porém, destaca-se que apenas o uso dessas técnicas não é o suficiente para mitigar todos os males da produção de petróleo e do esgotamento das reservas; outras medidas podem e devem ser tomadas, simultaneamente, para aumentar a eficiência e a rentabilidade da produção, tais como:

\begin{itemize}
\item \textbf{Exploração de reservas não convencionais} --- Xistos e folhelhos betuminosos, por exemplo, acumulam grandes quantidades de óleo. Várias dessas reservas já foram encontradas em regiões como Athabasca, no Canadá, cinturão do Orinoco, na Venezuela, e o Colorado, nos Estados Unidos. O custo de produção nessas reservas é considerável, mas já se projetam meios tecnológicos para reduzir o mesmo. Entre outras reservas não convencionais de hidrocarbonetos, há a presença de gás natural em solução existente na água de aquíferos; embora a razão de solubilidade do gás natural na água normalmente seja pequena, o imenso volume dos aquíferos perimitiria uma produção de grandes volumes desse gás. Uma outra reserva não convecional poderá ser o gás natural proveniente de hidratos localizados no fundo de oceanos e em regiões congeladas da Terra.
\item \textbf{Estimulação de Poços} De acordo com Thomas \textit{et al.}, a estimulação de poços é um conjunto de atividades realizadas com o objetivo de aumentar o índice de produtividade ou injetividade do poço \cite[p.~166]{engpetro}. Os principais métodos de estimulação são: fraturamento hidráulico, em que se cria, através de uma ruptura na rocha-reservatório causada por um elevado gradiente de pressão, um caminho preferencial de alta condutividade, facilitando um fluxo de fluidos do reservatório ao poço (ou vice-versa); e acidificação, onde se injeta um ácido com pressão inferior à pressão de fraturamento da formação, visando remover danos da mesma. Tais métodos contribuem para a aceleração da produção e até, em alguns casos, o aumento da eficiência de recuperação. A aplicação de métodos de estimulação pode, inclusive, ser feita em campos submetidos a operações de recuperação secundária.
\item \textbf{Uso de poços especiais} --- Nas últimas décadas houve um incremento considerável no uso dos chamados \textit{poços especiais}, que possuem como característica marcante a não-verticalidade, Segundo Thomas \textit{et al.}, esses poços são perfurados com várias finalidades, como: controlar um poço em \textit{blowout} por meio de poços de alívio; atingir formações produtoras abaixo de locais inacessíveis, como rios, lagos, cidades, entre outros; desviar a trajetória do poço de acidentes geológicos, como domos salinos e falhas; perfurar vários poços de um mesmo ponto, como é o caso da produção em plataformas marítimas; e desviar poços que tiveram seu trecho final perdido por problemas operacionais \cite[p.~106]{engpetro}. O uso desses poços inclinados, horizontais, multilaterais, etc., pode aumentar a velocidade de drenagem do reservatório, ou seja, antecipar a produção, bem como aumentar a eficiência de recuperação através do aumento da eficiência de varrido, por exemplo.
\item \textbf{Extração de líquidos de gás natural} --- A produção de hidrocarbonetos líquidos pode ser aumentada pela instalação de plantas de gasolina natural e de unidades portáteis de extração de líquidos de gás natural.
\item \textbf{Reestudo de áreas julgadas improdutivas ou antieconômicas} --- Mesmo que as reservas mundiais de petróleo sejam limitadas, elas estão longe de terem sido totalmente exploradas; de fato, apenas uma pequena porcentagem da superfície do planeta foi inteiramente explorada. Seja na terra ou no fundo do mar, há ainda perspectivas notáveis fora das áreas hoje em produção; além disso, as estimativas do volume de óleo que ainda poderá ser descoberto são ainda vagas. É com essa perspectiva que a indústria pode medir as oportunidades que tem à frente no caso de esgotamento das áreas hoje em produção. Portanto, de um modo geral, deve-se pensar sempre na adoção das seguintes medidas, sem danos ao andamento das operações de recuperação secundária: estudar novas áreas; estudar formações mais profundas (o pré-sal é um exemplo); reestudar áreas consideradas esgotadas ou de produção antieconômica; e investir mais dinheiro, tempo e pessoal em treinamento e pesquisa, visando melhorar os métodos de exploração e produção existentes.
\end{itemize}

\subsection{Classificação dos Métodos de Recuperação Secundária}

\subsection{Métodos Convencionais}

\subsection{Eficiência de Recuperação}

\subsection{Aspectos Operacionais da Injeção de Água}